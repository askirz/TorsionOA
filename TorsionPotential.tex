\documentclass[12pt]{article}
\usepackage{amsmath,amsthm,latexsym,amssymb,amsfonts,epsfig} 
%------------------
%--------- Definitions
%------------------
\def\be{\begin{equation}}
\def\ee{\end{equation}}
\def\ba{\begin{eqnarray}}
\def\ea{\end{eqnarray}}
\def\md{{\mathrm{d}}}
%\def\mx{{\mathrm{x}}}
\def\my{{\mathrm{y}}}
\def\mz{{\mathrm{z}}}
\def\l{\left}
\def\r{\right}
\def\ll{\left\{}
\def\rl{\right\}}
\def\lc{\left[}
\def\rc{\right]}
\def\lp{\left(}
\def\rp{\right)}
\def\la{\left\langle}
\def\ra{\right\rangle} 
 

%------------------
%--------- Title Page
%------------------
\title{ Torsion potential formalism.}

\author{{\sf Aureliano Skirzewski}\thanks{{\tt
askirz@gmail.com, skirz@ula.ve}}
\\
{\small Centro de Astrofisica Teorica, } \\
{\small Universidad de los Andes, 5101 Merida, Venezuela.} \\ 
\and 
{\sf Oscar Castillo-Felisola}\thanks{{\tt
o.castillo.felisola@gmail.com}}
\\
{\small UTFSM and CCTVal, CHILE} }

\date{}

%------------------
%--------- Document
%------------------
\begin{document}

\maketitle

%--------- Abstract
\begin{abstract}
We use torsion definition to explore new representation of
torsion as the field strength of it's gauge field potential.
\end{abstract}
%--------- Scattered Ideas
Formally, torsion is defined through 
\be 
\label{def1}
\l[\nabla _\mu,\nabla_\nu\r]f(x)=T_\mu{}^\rho{}_\nu
\partial_\rho f(x),
\ee
thus, we propose to interpret $ \partial_\rho $ as the local 
group generators, leading us to the conclusion that the 
covariant derivative  $\nabla_\mu$ on vector fields 
must be modified by a new gauge field 
expanded in it's generators $\gamma_\mu^\nu\partial_\nu$ as 
\be\nabla_\mu A^\nu \to D_\mu A^\nu=\partial_\mu A^\nu+\Gamma_
\mu{}^\nu{}_\rho A^\rho+\gamma_\mu^\rho \partial_\rho A^\nu,\ee
however, it's addition introduces modifications to the explicit 
relationship of Christopher's symbols and spacetime's 
curvature that we would like to avoid through the introduction 
of 
$\xi_\mu^\nu=\delta_\mu^\nu+
\gamma_\mu^\nu $ and the substitution
$\Gamma_\mu{}^\nu{}_\rho\to \xi_\mu^\sigma\Gamma_\sigma{}^\nu
{}_\rho$, to leave a covariant derivative defined as 
\be D_\mu A^\nu=\xi_\mu^\sigma\l(\partial_
\sigma A^\nu+\Gamma_\sigma{}^\nu {}_\rho A^\rho \r)\ee
which, if we substitute in (\ref{def1}),
\ba 
\l[D_\mu,D_\nu\r]f(x)
&=&\l[ \xi_\mu^\sigma\l(\delta_\nu^
\rho\partial_\sigma- \Gamma_\sigma{}^\rho{}_\nu 
\r)\xi_\rho^\lambda\partial_\lambda- \xi_\nu^\sigma\l(\delta_
\mu^\rho\partial_\sigma- \Gamma_\sigma{}^\rho{}_\mu 
\r)\xi_\rho^\lambda\partial_\lambda \r]f(x) \nonumber\\ &=& \l[ 
\xi_{[\mu|}^\sigma\l(\partial_\sigma\xi_{|\nu]}^\lambda-
\Gamma_\sigma{}^\rho{}_{|\nu]}\xi_\rho^\lambda+\Gamma_\sigma{}^
\lambda{}_{\rho}\xi_{|\nu]}^\rho\r) - 
\xi^\sigma_{[\mu|}\Gamma_\sigma{}^\lambda{}_{\rho}\xi_{|\nu]}^
\rho\r]\partial_\lambda f(x) \\ &=& \l[D_{[\mu} \xi_{\nu]}^\rho-
\xi_\mu^\lambda\Gamma_{[\lambda}{}^\rho{}_{\sigma]}\xi_\nu^\sigma 
\r]\partial_\rho f(x)\ea 

By definition, torsion should be acknowledged as a field strength 
tensor for the $\xi_\nu^\rho$ gauge potential plus the 
antisymmetric part of 
$ \Gamma_{[\mu}{}^\rho{}_{\nu]} $
\be\label{torsion} T_\mu {}^\lambda {}_\nu = D_{[\mu}\xi_{\nu]}
^\lambda -\xi_\mu^\rho\Gamma_{[\rho}{} ^\lambda{}_{\sigma]}\xi_\nu
^\sigma.\ee

Couplings to $\xi_\mu^\nu$, occurs for every field once we 
take its covariant derivative. In fact, this covariant
derivative could be gauged with any local symmetry with
generators $(t_m)^a_b$ through
\be D_\mu f^{a}_\nu=\xi_\mu^\lambda\partial_\lambda f^{a}
_\nu-\xi_\mu^\lambda\Gamma_\lambda{}^\sigma{}_\nu f^{a}_\sigma
+ A_\mu^m (t_m)^a_b f^{b}_\nu \ee

In order to study the modifications to the space of 
solutions of this modified general relativity, we will start
setting as Lagrangian of the system the usual Ricci scalar.
Since Ricci scalar is changed, following the usual definition 
\be\label{curvdef}\lc \nabla_\mu,\nabla_\nu\rc \omega_\lambda=
-R_{\mu\nu}{}^\rho{}_\lambda \omega _\rho,\ee
in order to proceed we  define $\nabla_\nu$ through 
\be D_\mu=\xi_\mu^\nu\nabla_\nu+ A_\mu^m t_m\ee and distinguish 
the contorsion tensor $\mathcal K_\mu{}^\lambda{}_\nu$ from the
metric affine connection $\hat\Gamma_\mu{}^\lambda{}_\nu$ 
such that  $$\Gamma_\mu{}^\lambda{}_\nu=\hat
\Gamma_\mu{}^\lambda{}_\nu+ \mathcal K_\mu{}^\lambda{}_\nu$$
and $\nabla_\mu\omega_\nu= \hat\nabla_\mu\omega_\nu- {\mathcal K}
_\mu{}^\lambda{}_\nu\omega_\lambda $
thus, (\ref{torsion}) is rewritten as \be T_\mu {}^\lambda {}_\nu = 
\xi_{[\mu|}^\rho\hat\nabla_\rho\xi_{|\nu]}^\lambda \ee 
and (\ref{curvdef}) 
\ba \l[\nabla_\rho,
\nabla_\sigma\r]\omega_\lambda&=&
\hat\nabla_{[\rho}\l( \hat\nabla_{\sigma]} 
\omega_\lambda-{\mathcal K}_{\sigma]}{}^\kappa{}_\lambda
\omega_\kappa\rp-{\mathcal K}_{[\rho}{}^{\tau}{}_{\sigma]}
\l( \hat\nabla_{\tau} \omega_\lambda-{\mathcal 
K}_{\tau}{}^\kappa{}_\lambda\omega_\kappa\r) \\ && - {\mathcal 
K}_{[\rho|}{}^{\tau}{}_{\lambda}\l( \hat\nabla_{|\sigma]} 
\omega_\tau-{\mathcal K}_{|\sigma]}{}^\kappa{}_{\tau}
\omega_\kappa\rp \\ &=& -\hat R _{\rho\sigma\lambda}{}^\tau\omega
_\tau-{\mathcal K}_{[\rho}{}^{\tau}{}_{\sigma]} 
\hat\nabla_{\tau} \omega_\lambda \\ \nonumber && - 
\l(\hat\nabla_{[\rho}{\mathcal K}_{\sigma]}{}^\kappa{}_\lambda-
{\mathcal K}_{[\rho}{}^{\tau}{}_{\sigma]} 
{\mathcal K}_{\tau}{}^\kappa{}_\lambda - 
{\mathcal K}_{[\rho|}{}^{\tau}{}_{\lambda}{
\mathcal K}_{|\sigma]}{}^\kappa{}_{\tau}
\r)\omega_\kappa \ea 
then, 
\ba \l[ D_\mu,
D_\nu\r] \omega_\lambda&=& \xi_{\mu}^\rho\nabla_{[\rho|} 
\l(\xi_{\nu}^\sigma\nabla_{|\sigma]}\omega_\lambda\r) \\ &=& 
\l(\xi_{[\mu|}^\rho\nabla_{\rho} \xi_{|\nu]}^\sigma\r)\nabla_{
\sigma}\omega_\lambda+\xi_{\mu}^\rho \xi_{\nu}^\sigma\l[\nabla_{
\rho},\nabla_{\sigma}\r]\omega_\lambda
\\ &=& \l(\xi_{[\mu|}^\rho\hat\nabla _{\rho} \xi_{|\nu]}^\sigma\r)
\nabla_{\sigma}\omega_\lambda-\xi_{\mu}^\rho 
\xi_{\nu}^\sigma\hat R_{\rho\sigma\lambda}{}^\tau
\omega_\tau \\ \nonumber && - \xi_{\mu}^\rho \xi_{\nu}^\sigma 
\l(\hat\nabla_{[\rho}{\mathcal K}_{\sigma]}{}^\tau{}_\lambda +
{\mathcal K}_{[\rho|}{}^\tau{}_{\kappa} 
{\mathcal K}_{|\sigma]}{}^{\kappa}{}_{
\lambda}\r)\omega_\tau \\ \nonumber &=& 
\l(\xi_{[\mu|}^\rho\hat\nabla _{\rho} \xi_{|\nu]}^\sigma\r)
\hat\nabla_{\sigma}\omega_\lambda-\l[ \xi_{[\mu|}^\rho\hat\nabla 
_{\rho} \xi_{|\nu]}^\sigma 
{\mathcal K}_{\sigma}{}^\tau{}_\lambda \r. 
\\ && \nonumber \l. +\xi_{\mu}^\rho 
\xi_{\nu}^\sigma\l(\hat R_{\rho\sigma\lambda}{}^\tau + 
\hat\nabla_{[\rho}{\mathcal K}_{\sigma]}{}^\tau{}_\lambda +
{\mathcal K}_{[\rho|}{}^\tau{}_{\kappa} 
{\mathcal K}_{|\sigma]}{}^{\kappa}{}_{
\lambda}\r)\r]\omega_\tau ,\ea 
parting from the last expression we can build up a scalar that
plays the role of Lagrangian of the system. In fact, there are in principle 

\ba {\mathcal L}= \sqrt{g}g^{\mu\nu}\ea

\end{document}
